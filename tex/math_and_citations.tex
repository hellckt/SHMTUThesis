\chapter{数学符号与引用文献的标注}

\section{数学}

\subsection{数字与单位}

宏包 \pkg{siunitx} \footnote{\url{http://tug.ctan.org/macros/latex/exptl/siunitx/siunitx.pdf}}提供了更好的数字和单位支持,具体请查看相关文档:
\begin{itemize}
	\item \num{12345,67890}
	\item \num{1+-2i}
	\item \num{.3e45}
	\item \num{1.654 x 2.34 x 3.430}
	\item \si{kg.m.s^{-1}}
	\item \si{\kilogram\metre\per\second}
	\item \si[per-mode=symbol]{\kilogram\metre\per\second}
	\item \si[per-mode=symbol]{\kilogram\metre\per\ampere\per\second}
	\item \numlist{10;20;30}
	\item \SIlist{0.13;0.67;0.80}{\milli\metre}
	\item \numrange{10}{20}
    \item \SIrange{10}{20}{\degreeCelsius}
	\item \SIrange{0.13}{0.67}{\milli\metre}
	\item \ang{10}
	\item \ang{12.3}
	\item \ang{4,5}
	\item \ang{1;2;3}
	\item \ang{;;1}
	\item \ang{+10;;}
	\item \ang{-0;1;}
\end{itemize}

\subsection{数学符号和公式}

本小节仅演示基本用法,数学符号、公式、数组的详细内容,请查看文档\footnote{\url{https://en.wikibooks.org/wiki/LaTeX/Mathematics}}。

微分符号 $\dif$ 应使用正体,本模板提供了 \cs{dif} 命令。除此之外,模板还提供了一些命令方便使用:
\begin{itemize}
  \item 圆周率 $\uppi$:\verb|\uppi|
  \item 自然对数的底 $\upe$:\verb|\upe|
  \item 虚数单位 $\upi$, $\upj$:\verb|\upi| \verb|\upj|
\end{itemize}

公式应另起一行居中排版。公式后应注明编号,按章顺序编排,编号右端对齐。

\begin{equation}
	\cos (2\theta) = \cos^2 \theta - \sin^2 \theta
\end{equation}

\begin{equation}
  \frac{\dif^2 u}{\dif t^2} = \int f(x) \dif x.
\end{equation}

公式末尾是需要添加标点符号的,至于用逗号还是句号,取决于公式下面一句是接着公式说的,还是另起一句。

\begin{equation}
	\frac{2h}{\pi}\int_{0}^{\infty}\frac{\sin\left( \omega\delta \right)}{\omega}
	\cos\left( \omega x \right) \dif\omega = 
	\begin{cases}
		h, \ \left| x \right| < \delta, \\
		\frac{h}{2}, \ x = \pm \delta, \\
		0, \ \left| x \right| > \delta.
	\end{cases}
\end{equation}

公式较长时最好在等号“$=$”处转行。
\begin{align}
    & I (X_3; X_4) - I (X_3; X_4 \mid X_1) - I (X_3; X_4 \mid X_2) \nonumber \\
  = & [I (X_3; X_4) - I (X_3; X_4 \mid X_1)] - I (X_3; X_4 \mid \tilde{X}_2) \\
  = & I (X_1; X_3; X_4) - I (X_3; X_4 \mid \tilde{X}_2).
\end{align}

如果在等号处转行难以实现,也可在 $+$、$-$、$\times$、$\div$运算符号处转行,转行时运算符号仅书写于转行式前,不重复书写。
\begin{multline}
  \frac{1}{2} \Delta (f_{ij} f^{ij}) =
    2 \left(\sum_{i<j} \chi_{ij}(\sigma_{i} - \sigma_{j})^{2}
    + f^{ij} \nabla_{j} \nabla_{i} (\Delta f) \right. \\
  \left. + \nabla_{k} f_{ij} \nabla^{k} f^{ij} +
    f^{ij} f^{k} \left[2\nabla_{i}R_{jk}
    - \nabla_{k} R_{ij} \right] \vphantom{\sum_{i<j}} \right).
\end{multline}


需要在文中引用某个指定公式,如公式~\ref{eq:array}所示:

\begin{equation}
	A_{m,n} = 
	\begin{pmatrix}
		a_{1,1} & a_{1,2} & \cdots & a_{1,n} \\
		a_{2,1} & a_{2,2} & \cdots & a_{2,n} \\
		\vdots  & \vdots  & \ddots & \vdots  \\
		a_{m,1} & a_{m,2} & \cdots & a_{m,n} 
	\end{pmatrix}\label{eq:array}
\end{equation}

\subsection{定理环境}

示例文件中使用 \pkg{ntheorem} 宏包配置了定理、引理和证明等环境。用户也可以使用\pkg{amsthm} 宏包。

这里举一个“定理”和“证明”的例子:
\begin{theorem}[留数定理]\label{thm:res}
  假设 $U$ 是复平面上的一个单连通开子集,$a_1, \ldots, a_n$ 是复平面上有限个点,
  $f$ 是定义在 $U \backslash \{a_1, \ldots, a_n\}$ 上的全纯函数,如果 $\gamma$
  是一条把 $a_1, \ldots, a_n$ 包围起来的可求长曲线,但不经过任何一个 $a_k$,并且
  其起点与终点重合,那么:

  \begin{equation}
    \label{eq:res}
    \ointop_\gamma f(z)\, \dif z = 2\uppi \upi \sum_{k=1}^n \operatorname{I}(\gamma, a_k) \operatorname{Res}(f, a_k).
  \end{equation}

  如果 $\gamma$ 是若尔当曲线,那么 $\operatorname{I}(\gamma, a_k) = 1$,因此:

  \begin{equation}
    \label{eq:resthm}
    \ointop_\gamma f(z)\, \dif z = 2\uppi \upi \sum_{k=1}^n \operatorname{Res}(f, a_k).
  \end{equation}

  在这里,$\operatorname{Res}(f, a_k)$ 表示 $f$ 在点 $a_k$ 的留数,
  $\operatorname{I}(\gamma, a_k)$ 表示 $\gamma$ 关于点 $a_k$ 的卷绕数。卷绕数是
  一个整数,它描述了曲线 $\gamma$ 绕过点 $a_k$ 的次数。如果 $\gamma$ 依逆时针方
  向绕着 $a_k$ 移动,卷绕数就是一个正数,如果 $\gamma$ 根本不绕过 $a_k$,卷绕数
  就是零。
\end{theorem}

定理~\ref{thm:res} 的证明。
	
\begin{proof}
  首先,由\dots

  其次,\dots

  所以,\dots
\end{proof}

\section{引用文献的标注}

按照上海海事大学的要求,参考文献外观应符合国标 GB/T 7714 的要求。具体建议使用87版标准,由于这个版本太老(1988年1月1日实施),故本模版使用该国标下最新的2015版标准。本模版使用 \BibLaTeX\ 配合 \pkg{biblatex-gb7714-2015} 样式包\footnote{\url{https://www.ctan.org/pkg/biblatex-gb7714-2015}}控制参考文献的输出样式,后端采用 \pkg{biber} 管理文献。

请注意 \pkg{biblatex-gb7714-2015} 宏包 2016 年 9 月才加入 CTAN,如果你使用的\TeX\ 系统版本较旧,可能没有包含 \pkg{biblatex-gb7714-2015} 宏包,需要手动安装。\BibLaTeX\ 与 \pkg{biblatex-gb7714-2015} 目前在活跃地更新,为避免一些兼容性问题,推荐使用较新的版本。

正文中引用参考文献时,使用 \verb|\cite{key1,key2,key3...}| 可以产生“上标引用的参考文献”。使用\verb|\parencite{key1, key2, key3...}| 则可以产生水平引用的参考文献。建议将bibtex文献中的标示都改为英文,以免出现不兼容现象。

具体请看下面的例子,将会穿插使用水平的和上标的参考文献:Chen调查了用于语言n-gram建模的平滑模型的最广泛使用的算法,并提出了改进的语言模型平滑度,从而改善了语音识别性能\cite{chen1999empirical};SRILM是C ++库,可执行程序和帮助程序脚本的集合,旨在允许为语音识别和其他应用程序生成统计语言模型并进行实验\cite{stolcke2002srilm}。Sundermeyer、Soutner、王毅、梁军等人将LSTM应用到自然语言处理领域,并获得了不错的实验结果\cite{sundermeyer2012lstm, soutner2013application, wangyi2018, liangjun2015}。文献\parencite{sundermeyer2012lstm, soutner2013application, wangyi2018, liangjun2015}中均使用LSTM神经网络架构。

当需要将参考文献条目加入到文献表中但又不在正文中引用,可以使用\verb|\nocite{key1,key2,key3...}|。或者使用 \verb|\nocite{*}| 将参考文献数据库中的所有条目加入到文献表中。\nocite{*}
